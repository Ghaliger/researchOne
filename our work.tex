\documentclass[10pt,letterpaper]{article}
\begin{document}
\title{USING MOBILE APPLICATION TO CUB JUVENILE DELINQUENCY }

\maketitle
\section{Problem statement}
A very interesting definition in the dictionary portrays a child as immature and irresponsible. This implies that parents and the community in general is supposed to be vigilant in monitoring the decisions made by children and model the into responsible human beings. 
However, parents in suburbs spend more time away from these children and priotise things like jobs over their responsibility in modelling their children’s lives and decisions; often times parents live this responsibility to maids which exposes them to a lot more freedom than what is good for them. 
In response to this problem, our study prepossesses to provide an option for curbing juvenile delinquency by using technology in form of a mobile application. We propose to use a mobile application because it places the responsibility of watching over these children in the hands of responsible citizens of the society who notify authorities like the police in case a child is seen participating in such illegal activity. The citizen is offered an opportunity to either place a call directly to the police, send an email describing the situation or take a picture and send it as an attachment of the email. The application would of course have a GPS factor which would reveal the location of any of these notifications to the said authority.

\section{main objectives}
To identify the need to intervene and do something about the right of juvenile delinquency in Kampala. 
To develop a mobile application to cub juvenile delinquency in Kampala.
To establish a technological means to analyse the rate of this behaviour around various parts of Kampala.

\section{specific objectives}
To identify the need to intervene and do something about the rate of juvenile delinquency in Kampala.  
\end{document}